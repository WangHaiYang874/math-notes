\section{基本知识}

\subsection{介绍}

例子:最短路径,最短时间,等周不等式,都可以写成形式:
\[J[y] = \int_a^b F(x,y,y')dx\]

这类函子都有局部化的性质:可以拆成多段的小积分,局部求极值。

要继续下去,需要给出函数空间上恰当的距离,上述函子中的\(y\)
是恰当地属于\(C^1\)的。所以我们可以考虑距离:
\(|y| =\max|y| + \max|y'|\)。
并且假定我们所要找的解就在这个空间中。

除此之外,函子也得需要有连续性,即\(J:C^1([a,b])\to \R\)是个连续函数。
我们也需要线性函子的概念,见教科书第八页。

\begin{lem}
    若 \(\alpha(x)\) 连续,
    且\(\int_a^b\alpha(x)h(x)dx=0,\forall h\in C^1, h(a)=h(b)=0\).

    那么\(\alpha  \equiv 0\).
\end{lem}

\begin{lem}基本引理

\begin{itemize}
\item 若 \(\alpha(x)\) 连续,
    且\(\int_a^b\alpha(x)h'(x)dx=0,\forall h\in C^1, h(a)=h(b)=0\).

    那么\(\alpha  \equiv C\), 一个常数

\item 若 \(\alpha(x)\) 连续,
    且\(\int_a^b\alpha(x)h''(x)dx=0,\forall h\in C^1, h(a)=h(b)=0\).

    那么\(\alpha  \equiv Cx + D\), 俩个常数

\item 若\(\alpha,\beta\)为连续函数,
    且\(\int_a^b \alpha(x)h(x)+\beta(x)h'(x)dx = 0\).

    那么\(\beta\)可微,且\(\beta' = \alpha\).

\end{itemize}
\end{lem}


现在开始考虑变分. 对于定义在有距离的线性函数空间上的\(J[y]\)
令
\[\Delta J[h] = J[y+h]-J[y].\]
若有线性函子\(\phi\)和在零点处无穷小阶函子\(\epsilon\)使得
\(\Delta J[h] = \phi[h] + \epsilon|h|\),那么称\(\phi\)为
\(J\)的变分,记作\(\partial J\).
容易得知,变分若存在必唯一.

\begin{thm}
    \(J[y]\)在\(\hat y\)处取极值的必要条件是
    \[\partial J[h] = 0, \forall h\].
\end{thm}

从物理的角度来说,我们定义\(p = F_{y'}, H = -F + y'p\), 其意义分别为动量和能量(Hamiltonian算子)

\subsection{欧拉-拉格朗日方程}

在固定端点的变分问题里,令\(F\)连续可微,
若\(y\)是满足上一定理条件的极值曲线.
很容易有:
\[F_y - \frac{d}{dx}F_{y'} = 0\]
但这样做是相当于假定了极值曲线是平滑的.稍后会介绍处理不平滑曲线的方法

\begin{thm} Bernstein 微分方程解的存在性定理.见16页

    若函数\(F,F_y,F_{y'}\)都在有限区域内连续,
    且存在\(k>0\)和\(\alpha(x,y),\beta(x,y) \ge 0\)
    满足\(F_y > k, |F| \le \alpha y'^2 + \beta\)

    那么微分方程\(y'' = F(x,y,y')\)在给定不同端点的条件下有唯一解
\end{thm}


\begin{thm} E-L方程的特殊情况汇总

    \begin{itemize}
        \item 若\(F\)不直接取决于\(y\), 那么E-L公式推出:
        \(F_{y'} = C\).
        \item 若\(F\)不直接取决于\(x\), 那么E-L公式推出:
        \(F - y'F_{y'} = C\).
        \item 若\(F\)不直接取决于\(y'\), 那么E-L公式推出:
        \(F \equiv C\).
    \end{itemize}

\end{thm}

\subsection{多变量情况, 这节不用看}.

\begin{lem}基本引理
    若\(\alpha(x,y)\)在区域\(R\)上连续且满足:
    \[\int\int_R \alpha(x,y)h(x,y)dxdy \equiv 0, \forall h\]
    \(h\)在\(R\)上连续且\(h\)在\(R\)的边界\(\Gamma\)上为0. 那么
    \(\alpha = 0\).
\end{lem}

\(z = z(x,y)\).那么对于函子
\[J[z] = \int\int_R F(x,y,z,z_x,z_y)dxdy\].

有
\[\partial J = \int\int_R(F_z - \frac{\partial}{\partial x}F_{z_x} - \frac{\partial}{\partial y}F_{z_y} )h(x,y)dxdy\].
中间必须为零的那项是一个二阶PDE, 被称为欧拉等式. 可以用来解决:给定边界,求最小面积曲面.

\subsection{参数化}

把原有的方程改为
\[\int_{t_0}^{t_1} F(x(t),y(t),\frac{\dot y(t)}{\dot x(t)}) \dot x(t) dt = \int \Phi(x,y,\dot x,\dot y)dt\]
保留原方程性质的充要条件:\(\dot x, \dot y\)有一阶positive homogeneous,
也就是说\(\Phi(x,y,\lambda\dot x,\lambda\dot y) = \lambda \Phi(x,y,\dot x, \dot y)\).

值得注意的是(废话的),参数的选取不是本质的.

\subsection{高阶欧拉-拉格朗日方程}

取函子\(\int F(x,y_1,y_2,\dots,y_n,\dot y_1,\dots,\dot y_n)dx\).

其E-L方程为: \(F_{y} - \frac{d}{dx}F_{y'} + \frac{d^2}{dx^2}F_{y''} - \dots + (-1)^n \frac{d^n}{dx^n} F_{y^{(n)}}\)
