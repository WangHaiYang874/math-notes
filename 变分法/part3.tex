\section{典型与相关问题}

\subsection{基本介绍}

    典型, 也就是把做变量代换\((x,y,y') \to (x,y,p = F_{y'})\)是个
    非常有用的工具, 它可以把我们的EL条件化成形式较好的一阶微分方程,
    更有其妙用无穷. 同时, 我们还有转换\(F \to H\). 
    \[H = -F + y'p\]

    上述变量代换允许进行的条件为:
    \(\det\frac{\partial(p_1,\dots,p_n)}{\partial(y_1,\dots,y_n)} \ne 0\),
    或者更美好的条件是上述矩阵正定. 这样就有了原来的凸性. 

    而转换后的EL方程有如下形式:
    \begin{equation}
        \begin{aligned}
            dH &= - \frac{\partial F}{\partial x}dx - \sum \frac{\partial F}{\partial y_i}dy_i +\sum y_i' dp_i \\
            \frac{\partial H}{\partial x} &= -\frac{\partial F}{\partial x}, \\
            \frac{\partial H}{\partial y_i} &= -\frac{\partial F}{\partial y_i}, \\
            \frac{\partial H}{\partial p_i} &= y_i' \\
            \frac{dy_i}{dx} &= \frac{\partial H}{\partial p_i}, 
            \frac{dp_i}{dx} = \frac{\partial H}{\partial y_i}. 
        \end{aligned}
    \end{equation}
    只有最后一行是EL等式,前几行只是推导过程. 


\subsection{E-L的第一积分}

若\(F\)不直接与\(x\)相关, 那么有上一个式子, 我们知道 \(H\) 也和\(x\) 不直接相关.
\[
\frac{dH}{dx} 
    = \sum  \frac{\partial H}{\partial y_i}\frac{dy_i}{dx} 
        +   \frac{\partial H}{\partial y_i}\frac{dp_i}{dx} 
    = \sum  \frac{\partial H}{\partial y_i}\frac{\partial H}{\partial p_i}
        -   \frac{\partial H}{\partial p_i}\frac{\partial H}{\partial y_i}
    = 0
\]
所以\(H\)在每个满足EL方程的\(y\)上是常数, 如果我们假设所有物理系统里的运动都要服从
EL方程, 那么我们可以把\(H\)称作物理系统的第一积分, 也可以称作EL等式的第一积分. 

对于任意一个第一积分\(\Phi(y_1,\cdots,y_n, p_1,\cdots,p_n)\), 我们可以探索它的充要条件: 
\begin{equation}
    \frac{d\Phi}{dx} 
    = \sum  \frac{\partial \Phi}{\partial y_i}\frac{dy_i}{dx}
        +   \frac{\partial \Phi}{\partial p_i}\frac{p_i}{dx} 
    = \sum  \frac{\partial \Phi}{\partial y_i}\frac{\partial H}{\partial p+i}
        -   \frac{\partial \Phi}{\partial p_i}\frac{\partial H}{\partial y_i}
    = [\Phi,H]
\end{equation}
最后这个括号叫泊松括号, 由此我们可知第一积分的充要条件是\([\Phi,H]\)在轨迹上处处为零. 

\subsection{Legendre变换,典型变化}

由勒让德变换, 我们可以由另一种方式来获取EL方程的典型式. ...有点懒得介绍啊. 总而言之,
这是个involution,也就是变换俩次等于不变换. 

例子. \(f(\epsilon) = \frac{\epsilon^a}{a}, a>1\) 
可以被involution为\(H(p) = \frac{p^b}{b}, \frac{1}{a}+\frac{1}{b} = 1\). 

\begin{thm}基本定理
    对于下凸函数\(f(\epsilon_1, \cdots, \epsilon_n)\)和其勒让德变换\(H(p_1,\cdots,p_n)\), 有:
    \begin{itemize}
        \item $f(\epsilon) = \max_p[- H(p) + \epsilon p]$
        \item \(H(p) = \max_{\epsilon} -f(\epsilon) + \epsilon p\)
        \item (Young's 不等式)\(f(\epsilon) + H(p) \ge \epsilon p\)
        \item \(\alpha f(\epsilon) \to \alpha H(p/\alpha)\). 
        \item \(f(\alpha) \to H(p/\alpha)\)
        \item 叠加是个什么状态我忘了. 
        \item \(f(\epsilon - a) = H(p) + pa\)
    \end{itemize}
\end{thm}

有了勒让德变换的基本性质后, 我们能开始构造EL公式的典型了.
选取新的函子:
\[J[y] = \int_a^b -H(x,y,p) + py' dx.\]
就可以得到EL函数的典范型: 
\[\frac{\partial H}{\partial y} = - \frac{dp}{dx}, \quad \frac{\partial H}{\partial p} = \frac{dy}{dx}\]


\subsection{典型变换}

普通EL方程的不变算子有着形式: 
\[u = u(x,y), v = v(x,y), u_xv_y - u_yv_x \ne 0\]. 
那么典型变换(不改变EL典型方程的算子)有哪些呢? 

若\(Y_i = Y_i(x,y_1,\cdots, y_n,p_1,\cdots, p_n), P_i = P_i(x,y,p)\). 
那么有新的函子: 
\[J^*[Y_1,\cdots, Y_n, P_1,\cdots,P_n] = \int \sum P_iY_i' - H^* dx. \]
\begin{lem}
    若两个函子有相同的极值曲线, 那他俩间差一个全导数. 
\end{lem}
由此, 知道了充要条件: \(\sum p_idy_i - H dx = \sum P_idY_i - H^*dx + d\Phi(x,y,p)\). 
翻译以下, 就是: 
\[p_i = \frac{\partial \Phi}{\partial y_i},\quad P_i = \frac{{\partial \Phi}}{\partial Y_i},\quad H^* = H + \frac{\partial \Phi}{\partial x}\]
这个语境下, \(\Phi\)被称为生成函数. 生成函数还可以写成: \(\Psi = \Phi + \sum P_iY_i\).
这样就有了对偶
\[p_i = \frac{\partial \Psi}{\partial y_i},\quad Y_i = \frac{\partial \Psi}{\partial P_i},\quad H^* = H + \frac{\partial \Psi}{\partial x}\]

\subsection{Noether定理}

典型变换是一个空间里的对称, 而诺特定理完美的描述了这类对称能给我们带来什么样的结果: 各类守恒定理. 
最简单的情况就是在\(F\)与\(x\)无关时的第一积分\(H\). 而更一般的情况如下. 

考虑变换\(y(x)\to y^*(x^*)\): 
\[x^* = \Phi(x,y_1,\cdots,y_n,y'_1,\cdots,y'_n) = \Phi(x,y,y'),\quad y_i^* = \Psi_i(x,y_1,\cdots,y_n,y'_1,\cdots,y'_n) = \Psi_i(x,y,y')\]
我们称这个函子\(J[y]\)在上述变换下不变当且仅当
\[\int_{x_0^*}^{x_1^*}F(x^*,y^*,\frac{y^*}{x^*}) dx = \int_{x_0}^{x_1} F(x,y,\frac{dy}{dx})dx.\]
 若我们有一组由实数index的连续不变变换\(\Phi^\epsilon, \Psi^\epsilon\),
 且\(\Phi^0(x,y,y') = x, \Psi_i^0(x,y,y') = y_i\), 就有:  
\begin{thm}(Noether)
    \[\sum F_[y_i']\psi_i + (F - \sum y_i'F_{y_i'})\phi = \text{const}\]
    这里的\(\psi = \frac{\partial \Psi}{\partial\epsilon}, \quad \phi = \frac{\partial |Phi}{\partial\epsilon}\)
\end{thm}

这个定理的特殊情况就是我们之前说的\(H\)在不直接取决与\(x\)时, \(H\)是个第一积分. 

\subsection{最小动作原理}

把一个有\(n\)个粒子的物理系统的动能定义为:
\(T = \frac{1}{2} \sum m_i(\dot x_i^2 + \dot y_i^2 + \dot z_i^2)\).

再假设我们有动能:
\(U = U(t,x_1,y_1,z_1,\cdots,x_n,y_n,z_n)\). 
那么各个例子在各个方向上的力可以被定义为:
\(X_i = -\frac{\partial U}{\partial x_i}, Y_i = \frac{\partial U}{\partial y_i}, \cdots\). 

如此, 我们就可以引入拉格朗日算子(动作): 
\[L = T - U.\]

在一个固定的时刻\(t_0\), 系统是固定的状态. 
而此后系统以\(x_i(t),y_i(t),z_i(t)\)的方式演变, 那么有:
\begin{thm}principal of least action

    \((x(t),y(t),z(t))\)是函子\(J[x,y,z] = \int_{t_0}^{t_1} L dt\)的极值曲线. 
\end{thm}
最小运动原理的有限制条件(在一个曲面上运动)时一样成立, 而且常常只能局部成立. 


\subsection{守恒量}

接上一小节,我们有典型变量: \(p_{ix} = \frac{\partial L}{\partial \dot x_i} = m_i\dot x_i\)和对\(y,z\)相同的等式. 
由此有:
\[H = -L + \sum p_{ix}\dot x_i = U - T + \sum m_i\dot x_i^2 = U + T. \]
得到了以下: 
\begin{thm}守恒定理

    \begin{itemize}
        \item 能量守恒: 
            
            如果给定系统守恒, 也就是说拉格朗日函数和时间无关, 那么能量守恒

        \item 动量守恒

            如果action函子在变换: \((x,y,z)\to (x+\epsilon,y,z)\)下保持不变,
            那么\(x-\)动量: \(P_x = \sum p_{ix}\)守恒. 同理可以获得对\(P_y, P_z\)的类似公式. 
            所有的动量都守恒被称为动量守恒.
            这里的变换可以看成平移变换
        \item 角动量守恒

            如果action函子在变换: 
            \((x_i,y_i,z_i)\to (x_i\cos(\epsilon) + y_i\sin(\epsilon), -x_i\sin(\epsilon) + y_i\cos(\epsilon), z_I)\)
            下保持不变. 那么绕\(z\)轴的角动量:\(\sum p_{ix}y_i - p_{iy}x_i \)守恒.
            更一般的, 所有角动量守恒可以写作: \(\sum p_i \times r_i\)守恒. 
            这里的变换可以看成旋转变换. 
            
    \end{itemize}
\end{thm}


\subsection{汉密尔顿-雅克布方程}

对在一个曲面\(S\)所有经过\(A,B\)两点的曲线\(y(x) = (y_1,\cdots,y_n)\), 和一个给定的
函子\(J[y] = \int_{x_0}^{x_1} F(x,y,y') dx\). 若有极值曲线\(y^*\), 那么\(S = J[y^*]\)
被称为测地线距离, \(y^*\)被称为测地线. 显而易见的\(S = S(A,B)\). 

例子. 费马原则指出, 光速沿着时间最短的路程前进, 那么\(F = \frac{\sqrt{\dot x^2 + \dot y^2 + \dot z^2}}{v}\). 

给定Lagrangian L, 两点间的测地线, 可以得出: \(dS = \partial J\). 
由此又可以得出: 
\[dS = \partial J = \sum p_idy_i -Hdx\]
\[\frac{\partial S}{\partial x} + H(x,y_1,\cdots,y_n, \frac{\partial S}{\partial y_1}, \cdots, \frac{\partial S}{\partial y_n}) = 0. \]
这就是哈密顿-雅克布方程, 它是一个PDE. EL典型方程是上述方程的特征系统(? 这是什么鬼??)

\begin{thm}
    
    给定上述方程的一组解(取决于参数\(\alpha\)) \(S = S(x,y_1,\cdots,y_n,\alpha_1,\cdots,\alpha_m), m\le n\), 
    那么每个微分\(\frac{\partial S}{\partial \alpha_i}\)都是欧拉典型方程的第一积分. 
\end{thm}

\begin{thm}
    给出上述方程的完整积分(一般解)\(S\), 并假设矩阵\(|\frac{\partial^2 S}{\partial \alpha_i \partial y_k}|\)
    的行列式不为零.
    取任意的常数\(\beta_i, i = 1,\cdots,n\). 
    
    那么由关系:
    \[\frac{\partial }{\partial \alpha_i}S(x,y,\alpha) = \beta_i\]
    所决定的函数\(y_i\). 
    与函数\(p_i = \frac{\partial}{\partial y_i}S\).

    给出了典型方程
    \[\frac{dy_i}{dx} = \frac{\partial H}{\partial p_i}, \frac{dp_i}{dx} = - \frac{\partial H}{\partial y_i}\]
    的一般解. 
\end{thm}

不知道上面这玩意该怎么解释啊?把问题转化成了一个几何问题?还是说什么啊?
\subsection{费马原则}

在一个对时间对称的物理系统里, 我们有比最小动作原理更特殊且更美妙的原则: 费马原则. 
它提供了将一个选取从点\(A\)到点\(B\)真正路径(不依赖于时间)的问题, 变成在黎曼曲面上
找测地线的问题. 

给定一个时间对称的\(L(q,\dot q)\)和相对应的\(H\), 考虑在极值曲线上的变分, 满足: 
\begin{itemize}
    \item 从极值曲线\(q^*\)到极值曲线\(q^* + \delta q\)
    \item \(H\)在俩个极值曲线上都是常数. 
    \item 第二条路径会和第一条路径同时离开\(A\)点, 但可能会在不同的时间\(t_B + \delta t_B\)到达\(B\).
    \item 函子是 \(S\)
\end{itemize}
这类变分被称为同能变分. 

有\(\delta S^* = -H_B\delta t_B = -E\delta t_B\). 

现在可以把重写\(S\)为
\[S[q] = \int_{t_A}^{t_B}(\sum p_i \dot q_i - H) dt = S_0[q] - \int_{t_A}^{t_B}Hdt.\]
\(S_0[q] = \int_{t_A}^{t_B}(\sum p_i \dot q_i) dt\) 被称为是截断动作. \(p_i= \frac{\partial L}{\partial \dot q_i}\). 

上面俩点加起来, 可以获得截断动作\(\partial S_0\)在现实路径上的同能变分为零. 这就是最基础的费马原理. 

要想应用费马原理, 我们要在截断动作中实现时间对称. 有时或许可以利用\(E\)的值来消灭掉时间, 有时又不行. 

例子, 当有类似这样的式子时:
\[H = \frac{1}{2}\frac{\sum_{ik}\alpha_{ik}(q)dq_idq_k}{dt^2} + V(q) = E\]
可以翻译成: 
\[\sqrt{\sum_{ik}\alpha_{ik}(q) dq_i dq_k} = \sqrt{2(E-V(q))}dt. \]
就可以使得我们的式子写成不涉及时间的形式: 
\[S_0[q] = \int \sqrt{2(E-V(q))} \sqrt{\sum_{ik}a_{ik}(q)dq_idq_k}.\]

相当于有黎曼测度: 
\[ds^2 = 2(E-V(q))\sum_{ik}a_{ik}(q)dq_idq_k\]
一个内积的形式. 

这个应该是和哈密顿雅可比等式有着一定的关联, 但是上节课我没听懂, 因为上上节课我没听. 