% 前言
\documentclass{article}
\usepackage[UTF8, heading=true]{ctex}
\usepackage[margin = 1in]{geometry}
\usepackage{amssymb}
\usepackage{amsmath}
\usepackage{amsfonts}

\usepackage{fancyhdr}
\pagestyle{fancy}
\fancyhf{}
\lhead{}
\rhead{}
\lfoot{}
\rfoot{}
\cfoot{\thepage}


% \usepackage{natbib}
\usepackage{graphicx}

\usepackage{hyperref}
\hypersetup{
    colorlinks=true,
    linkcolor=blue,
    filecolor=magenta,
    urlcolor=cyan,
}

% \usepackage{unicode-math}
% \setmathfont{XITS Math}
% \setmathfont[version=setB,StylisticSet=1]{XITS Math}

\newtheorem{problem}{Problem}[section]
\newtheorem{lemma}{Lemma}[section]
\newtheorem{theorem}{Theorem}[section]
\newtheorem{corollary}{Corollary}[section]
\newtheorem{remark}{Remark}[section]
\newtheorem{exercise}{练习}[section]
\newtheorem{defi}{定义}[section]

\newcommand{\Real}{\mathbb R}
\newcommand{\Z}{\mathbb Z}
\newcommand{\Complex}{\mathbb C}
\newcommand{\Q}{\mathbb Q}
\newcommand{\st}{\operatorname{s.t. }}
\newcommand{\obj}{\operatorname{obj}}
\newcommand{\cate}{\mathscr C}
\newcommand{\mor}{\operatorname{Mor}}

% \includeonly{}

\begin{document}

\title{Rising sea notes and solutions}
\author{王海阳Haiyang Wang}
\date{\today}
\maketitle

\begin{abstract}
    读Rising sea时的笔记, 和作业题的记录. 文件中的section为书中的chapter.  
    
    合作: ??? 暂无

    参考:
    rising sea

\end{abstract}

\tableofcontents

\section{简单范畴论}

\subsection{动机}
\begin{itemize}
    \item  用万有性定义乘积. 
        若存在, 那么在isomorphism的意义下唯一. 
    \item  定义阿贝尔范畴(todo)
\end{itemize}

\subsection{范畴和函子}
\begin{defi}
    一个(局部小)范畴\(\cate\)由: 
    \begin{itemize}
        \item 一组对象\(\obj(\cate)\)
        \item 每对对象 \(A,B\) 间的态射集\(\mor(A,B)\ni f\). 
        \item 态射的复合: 对\(f : A\to B, g: B\to C\), 有\(h = g\circ f \in \mor(A\to C)\)
        \item 
    \end{itemize}
\end{defi}

% \bibliographystyle{plain}
% \bibliography{references}
% \bibliography{}
\end{document}
